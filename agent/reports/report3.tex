\documentclass{jsarticle}

\usepackage{listings,jlisting}
\usepackage[dvipdfmx]{graphicx}
\usepackage{bmpsize}
\usepackage{bm}

\lstset{
    basicstyle={\ttfamily},
    identifierstyle={\small},
    commentstyle={\smallitshape},
    keywordstyle={\small\bfseries},
    ndkeywordstyle={\small},
    stringstyle={\small\ttfamily},
    frame={tb},
    breaklines=true,
    columns=[l]{fullflexible},
    numbers=left,
    xrightmargin=0zw,
    xleftmargin=3zw,
    numberstyle={\scriptsize},
    stepnumber=1,
    numbersep=1zw,
    lineskip=-0.5ex
}
\begin{document}

\title{計算機科学実験及演習4 エージェント 課題3}
\author{1029-28-2473 二見 颯}
\maketitle

\section{プログラム概要}
課題1で提出したハードマージンSVMに加え、ソフトマージンSVMを新しく実装した。
モデルの評価のために、K-fold交差検証を行えるようにして、
特にGaussカーネルSVMについて
予測精度が最大となる最適なハイパーパラメータの組を求めた。
データの正規化にも対応した。

\end{document}
